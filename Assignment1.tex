\documentclass{article}[12pt]

\title{Microelectronic Circuits Assignment 1}
\author{Sai Kartik, Manpreet Singh, Rajeev Rajagopal}
\date{February, 2022}

\begin{document}
\maketitle
\section*{Question 1}
Value of components (resistor/capacitor) present in the circuit:
\begin{center}
    \begin{tabular}{|c | c | c|}
        \hline
        Sl. No. & Component Name & Value       \\
        \hline
        1       & R1(k$\Omega$)  & 15k$\Omega$ \\ %last digit of first group member ID+10=5+10=15
        2       & C1 (nF)        & 1nF         \\ %0.1*(last digit of 3rd group member ID+tut sec number) = 0.1*(7+3)=1
        \hline
    \end{tabular}
\end{center}
\subsection*{Circuit as on LT SPICE}
%add circuit diagram
\subsection*{Graphs}
%add graphs: frequency response, input, output and transfer characteristics
\subsection*{Miscellaneous calculations}
%find DC operating point, small signal equivalent
\newpage
\section*{Question 2}
Value of components (resistors/gain) present in the circuit
\begin{center}
    \begin{tabular}{|c | c | c|}
        \hline
        Sl. No. & Component Name & Value       \\
        \hline
        1       & R1(k$\Omega$)  & 30k$\Omega$ \\ %(last digit of 3rd group member ID+3) * tut section = (7+3)*3 = 30
        2       & R2(k$\Omega$)  & 17k$\Omega$ \\ %product of last two digits of 1st group member ID + tut section = (3*5)+2=17
        3       & R3(k$\Omega$)  & 30k$\Omega$ \\ %sum of last three digits of 2nd group member ID % 8)* tut section = (4+1+9)%8*3=30
        4       & k              & 18          \\ %last 2 digits of 1st group member admission year - tut section = 20-2=18
        \hline
    \end{tabular}
\end{center}
\subsection*{Circuit as on LT SPICE}
%add circuit diagram
\subsection*{Z, Y and h Parameters}
%add Z, Y and h parameters
\subsection*{Calculations}
%add hand calculations of Z, Y and h parameters
\subsection*{Load resistance value at port 2}
%calculate according to question given
\end{document}